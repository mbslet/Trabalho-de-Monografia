\chapter{Introdução}
\addcontentsline{toc}{chapter}{Introdução}

Este capítulo tem o objetivo de apresentar uma contextualização sobre os assuntos tratados nesse trabalho, explorando as abordagens de reutilização de componentes e interfaces UI e as compreensões sobre User Interface Design. A partir desse dessa contextuaização pode-se entender a problemática tratada sobre o tema, dando origem então a questão de pesquisa a ser trabalhada nesse estudo e por conseguinte, a justificativa e os principais objetivos da monografia, finalizando o capítulo com a organização dos demais capítulos desse trabalho.

\section{Contextualização}
Segundo \cite{10.1145/2448963.2448965}, a implementação de interfaces de usuário usualmente é uma tarefa maçante podendo ser um tanto repetitiva. 

implementar interfaces de usuário costuma ser uma tarefa entediante. Para resolver isso, os Construtores de IU foram propostos para oferecer suporte à descrição de widgets, sua localização e sua lógica. Um aspecto ausente dos Construtores de IU é, no entanto, a capacidade de reutilizar e compor a lógica do widget. Em nossa experiência, isso leva a uma quantidade significativa de duplicação no código da IU. Para resolver esse problema, criamos o Spec: um UIBuilder para Pharo com foco na reutilização. Com Spec, as propriedades do widget são definidas declarativamente e anexadas a classes específicas conhecidas como classes composíveis. Uma classe combinável define sua própria descrição de widget, bem como a ponte modelo-widget e a lógica de interação de widget. Este artigo apresenta o Spec, mostrando como ele permite a reutilização perfeita de widgets e como eles podem ser personalizados. Depois de apresentar Spec e sua implementação, discutimos como seu uso no Pharo 2. 0 cortou pela metade a quantidade de linhas de código de seis de suas ferramentas, principalmente por meio da reutilização. Isso mostra que o Spec cumpre seus objetivos de permitir a reutilização e composição da lógica do widget.