\chapter{Suporte Tecnológico}

Este capítulo apresenta as ferramentas e tecnologias que foram e serão utilizadas no desenvolvimento deste trabalho. a seção de \nameref{sec:sg} aborda técnicas de gestão para essa monografia, como o versionamento e a metodologia kanban e ági. Por conseguinte,a seção do Desenvolvimento da Monografia, trazendo o ferramental utilizados para o apoio de pesquisa e a elaboração desta tese. E então, a seção de Desenvolvimento do Software, o qual será composto por estapas e ferramentas para desenvolver o modelo de reutilização de UI. Por fim, tem-se a seção referente ao Resumo do Capítulo

\section{Suporte e Gestão}
\label{sec:sg}

Para ter-se um bom controle e organização na construção desse trabalho, foi-se utilizado ferramentos de controle de versionamento como o \nameref{sec:git} e \nameref{sec:github}. Assim como ferramentas e técnicas das metodologias \nameref{sec:kanban} e \nameref{sec:agil}.

\subsection{Git}
\label{sec:git}

Git é um sistema de controle de versão distribuído disponível em todas as plataformas de desenvolvimento convencionais por meio de uma licença de software livre \cite{6188603}. Foi-se optado por esta ferramenta por conseguir gerir documentos, códigos e arquivos de forma que o armazenamento não seja perdido ou possa ser resgatado.

\subsection{GitHub}
\label{sec:github}

O GitHub é um repositório baseado na web para projetos de software e é declaradamente a maior comunidade de código aberto do mundo, hospedando mais de 31 milhões de repositórios que incluem o código e a documentação desse código. O GitHub inclui ferramentas de desenvolvimento como rastreamento de problemas, notificações, diferenciais e painéis de status \cite{7740497}. A Figura

\subsection{Kanban}
\label{sec:kanban}
a

\subsection{Ágil}
\label{sec:agil}
a















\section{Desenvolvimento}

O Desenvolvimento (Miolo ou Corpo do Trabalho) é subdividido em seções de 
acordo com o planejamento do autor. As seções primárias são aquelas que 
resultam da primeira divisão do texto do documento, geralmente 
correspondendo a divisão em capítulos. Seções secundárias, terciárias, 
etc., são aquelas que resultam da divisão do texto de uma seção primária, 
secundária, terciária, etc., respectivamente.

As seções primárias são numeradas consecutivamente, seguindo a série 
natural de números inteiros, a partir de 1, pela ordem de sua sucessão no 
documento.

O Desenvolvimento é a seção mais importante do trabalho, por isso exigi-se 
organização, objetividade e clareza. É conveniente dividi-lo em pelo menos 
três partes:

\begin{itemize}

	\item Referencial teórico, que corresponde a uma análise dos trabalhos 
	relevantes, encontrados na pesquisa bibliográfica sobre o assunto. 
	\item Metodologia, que é a descrição de todos os passos metodológicos 
	utilizados no trabalho. Sugere-se que se enfatize especialmente em (1) 
	População ou Sujeitos da pesquisa, (2) Materiais e equipamentos 
	utilizados e (3) Procedimentos de coleta de dados.
	\item Resultados, Discussão dos resultados e Conclusões, que é onde se 
	apresenta os dados encontrados a análise feita pelo autor à luz do 
	Referencial teórico e as Conclusões.

\end{itemize}

\section{Uso de editores de texto}

O uso de programas de edição eletrônica de textos é de livre escolha do autor. 

